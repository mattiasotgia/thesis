% !TEX root = ../main.tex

% \addchap{Abstract}

% \addsec{\@title}

\begin{abstract}
The three-flavor neutrino mixing minimal extension of the Standard Model (SM) has been established by a number of experiments in the past two decades. However, a series of experimental anomalies were observed, indicating a possible hint of the existence of a fourth neutrino, called \emph{sterile neutrino} because it does not undergo weak interaction.

This $3+1$ extension of the SM is the main physics target of the ICARUS experiment as part of the Short-Baseline Neutrino (SBN) program at Fermilab. The ICARUS-T600 760-ton detector is a Liquid Argon Time Projection Chamber (LAr-TPC) successfully employed at the LNGS laboratories for a three-year physics run and now collecting data at Fermi National Accelerator Laboratory (FNAL). The physics program of the ICARUS experiment also includes the measurement of neutrino-Argon cross sections employing the off-axis Neutrino at the Main Injector (NuMI) beam and several Beyond Standard Model studies.

The automatic TPC event reconstruction in ICARUS is performed using the Pandora Pattern Finding Algorithm framework that performs a 3D reconstruction of the image recorded in the collected event, including the identification of interaction vertices and the classification of tracks and showers inside the TPC.

In view of the standalone ICARUS oscillation $\nu_\mu$ analysis and of the future combined SBN oscillation analysis, a thorough evaluation of the performances of reconstruction chain, as well as the systematic uncertainties induced on the reconstructed neutrino energy spectrum is essential. The main objective of this work is to evaluate the performances of single steps of the reconstruction sequence, while possibly testing improvements of the machine learning algorithms employed in specific stages of the chain.
\end{abstract}
