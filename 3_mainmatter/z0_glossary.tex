\chapter{Acronyms}
\setcounter{page}{1}
\renewcommand{\thepage}{\Alph{chapter}--\arabic{page}}

Throughout the thesis, multiple acronyms or concepts are presented and used; here are the most common with some context

\begin{abbreviations}
    BNB & Booster Neutrino Beam, the main beam feeding the SBN experiment \\
    BSM & Beyond Standard Model \\
    CC/NC & Charge/neutral current interactions \\
    CKM & Cabibbo-Kobayashi-Maskawa matrix of quark mixing \\
    CNGS & Cern Neutrinos to Gran Sasso beam \\
    Coh. & Coherent neutrino scattering \\
    DaR/DiF & Decay-at-Rest/-in-Flight \\
    DIS & Deep inelastic scattering interaction \\
    DUNE & Deep Underground Neutrino Experiment \\
    ES & Elastic scattering interaction \\
    GALLEX \newline GALLEX+GNO & Gallium neutrino observatory at LNGS \\
    ICARUS \newline \emph{or SBN-FD} & \emph{Imaging Cosmic And Rare Underground Signals}, as it was called in the Gran Sasso era, now Far Detector in the SBN experiment \\
    LArTPC & Liquid Argon Time Projection Chamber \\
    LSND & Liquid Scintillator Neutrino Detector \\
    LEP & Large Electron Positron collider \\
    LNGS & Laboratori Nazionali del Gran Sasso \\
    Mini(Micro)BooNE & Mini(micro) Booster Neutrino Experiment \\ 
    MSW & Mikheyev-Smirnov-Wolfenstein oscillation of neutrinos in matter effect \\ 
    NO/NH \newline \emph{and} IO/IH & Normal and inverted ordering/hierarchy of neutrino masses \\
    NOvA & NuMI Off-axis $\PGne$ Appearance \\
    NuMI & Neutrinos at the Main Injector \\
    PMNS & Pontecorvo-Maki-Nakagawa-Sakata matrix of neutrino oscillation \\
    PMT & Photomultiplier tubes \\
    QE & Quasielastic interactions \\
    Res. & Resonant pion production interaction \\
    SAGE & Soviet-American Gallium experiment \\
    SBN & The Short Baseline Neutrino experiment, consisting of the three detectors on the BNB baseline \\
    SBND & Short Baseline neutrino Near Detector \\ 
    SK & Super Kamiokande \\ 
    SM & the Standard Model of particle physics \\
    SNO & Sudbury Neutrino Observatory \\
    S$\Pp\PAp$S & Super Proton (anti)Proton Synchrotron \\ 
\end{abbreviations}
