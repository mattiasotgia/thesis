% !TEX root=../main.tex

\addchap{Conclusions}\label{chap:conclusions}

\dictum[Anonymous]{I conclude that when I'm done with my thesis, life will be better}
\vspace{1em}

% STRUCTURE

%% Description 
In the past twenty years several short-baseline anomalies hinted at the possibility of a fourth eV-scale ``sterile'' neutrino state,  with incompatible results from appearance and disappearance results. The need to unravel the sterile neutrino picture led to the development of the SBN program. In the SBN program two functionally identical LArTPC-based experiments, the ICARUS and SBND experiments, are employed to measure with unprecedented ${>}5\sigma$ sensitivity short baseline neutrino oscillations, with an accumulated predicted exposure of \SI{6.6e20}{POT}. ICARUS was the first large-scale LArTPC ever built and was the largest LArTPC, with an active LAr mass of \SI{476}{\tonne}, until the development of the {ProtoDUNE} detector. It is installed at Fermilab at a distance of \SI{600}{\m} from the on-axis BNB neutrino beam source, collecting neutrinos also from the NuMI neutrino beam ${\sim}\SI{6}{\degree}$ off-axis. It started data taking in 2022 and has now completed its fourth data-taking campaign. SBND operates closer to the BNB beam source, at a distance of \SI{110}{\metre}, and joined ICARUS data taking as part of the SBN program, in 2024. SBND is relevant for the SBN program since it provides a flux monitor, measuring the unoscillated neutrino spectrum.

Data streams from both the ICARUS and SBND detectors are analysed to extract the physical information with a wide set of tools, performing multiple operations, from the raw signal processing to higher-level reconstruction of the interaction happening inside the LAr active volume. 

This thesis focuses on a thorough study of the high-level reconstruction of the interaction, with a primary focus on the Pandora framework. A large set of tools is involved in performing the event reconstruction, and different approaches are taken to ensure a robust result is achieved. One key task of the event reconstruction and signal identification pipeline is the identification of the interaction hierarchy from the processed signal deposited on the TPC wire-planes: all the final state particles in the interaction and their orientation in three-dimensional space are created building from their 2D projections on the readout planes, the interaction vertex is assigned, and the parent-daughter hierarchy is defined. 
In the context of the ICARUS experiment, two main frameworks are adopted: Pandora, which takes a more ``task-oriented'' approach to the event reconstruction task, providing hundreds of algorithms sequentially building up events, and SPINE, which provides an end-to-end optimisable machine-learning-based approach to the event reconstruction. 


The highly modular approach of the Pandora event reconstruction framework allows algorithms within the reconstruction chain to be modified and replaced. This is exploited by a set of tools that enable the use of true Monte Carlo information to alter the reconstruction output of the different stages. This comprehensive set of tools has already been tested and used in the context of the SBND reconstruction, as well as for other Pandora-based reconstruction pipelines \cite{Mawby:2023nws, Nguyen:2023_cheatingPandora}. These tools allow to investigate the impact of each step of Pandora reconstruction by altering the reconstructed objects using simulated events, where the true information derived from Monte Carlo is available. Using these altered reconstruction tools provides a strategy to validate the contribution of each algorithm to the entire reconstruction chain. 

The first part of the work was devoted to thoroughly validating the impact of using these algorithms to the event reconstruction chain in different places, either as a standalone replacement of the nominal algorithm or cumulatively, replacing the ``nominal'' reconstruction algorithms up to a certain step of the sequence. Using a sample of events corresponding to an exposure of \SI{8e19}{POT}, or about \num{6000} $\PGnGm$CCQE Np interaction $1\PGm N\Pp$ events inside the detector active volume, selected using the same event selection as the ongoing standalone ICARUS $\PGnGm$-disappearance study, each algorithm was characterised in terms of its impact on downstream variables. 

After a comprehensive validation of all these algorithms, testing their implementation in the context of the ICARUS event reconstruction, the results are consistent with expectations. We did find issues with the current implementation of the particle-hierarchy creation, which prevented us from using its cheated version in any of the subsequent studies. 

Having validated these tools, we developed a method to exploit them to extract the efficiencies of each stage of Pandora reconstruction chain. We compared five reconstruction configurations: a sequence of four ``cheated'' configurations, arranged hierarchically so that the first has all stages cheated, the second cheats all but the final stage, the third cheats only the first two stages, and the fourth cheats only the first stage; and finally, a baseline configuration using the standard, uncheated Pandora setup.

For all configurations, we evaluated the overall reconstruction and selection efficiency. To isolate the contribution of each reconstruction stage, we compared pairs of configurations: one in which all stages up to the step of interest were cheated, and the subsequent configuration in which that final stage was left uncheated.

To decouple the impact of the particle identification stage, which occurs downstream of event reconstruction, we designed a modified event selection. In this version, cuts related to particle identification were bypassed and true Monte Carlo labels were used instead. By comparing the events selected with the standard and modified procedures, we decoupled the particle identification efficiency for each of the five configurations from the efficiency of the stages.

The results of this analysis, showing the efficiency for all the major steps of the reconstruction chain, are presented below \begin{equation*}
    \begin{aligned}
        \epsilon_\mathrm{reco.} =&\
        \epsilon_\mathrm{2D\ clusters} &\times&\ 
        \epsilon_\mathrm{vertex\ creation} &\times&\ 
        \epsilon_\mathrm{3D\ reco.} &\times&\ 
        \epsilon_\mathrm{particle\ class.} =\\  
        =&\ \SI{89.7(1.8)}{\percent} &\times&\ 
        \SI{91.9(1.6)}{\percent} &\times&\ 
        \SI{97.7(1.6)}{\percent} &\times&\ 
        \SI{98.6(1.5)}{\percent}.
    \end{aligned}
\end{equation*}

The correct interpretation of our results requires some remarks. Firstly, we obtained them by targeting a specific event topology. Therefore, it is reasonable to assume that a different event topology would yield significantly different results. As explicitly stated throughout the text, we made numerous assumptions and approximations to obtain these results. We assumed that all correlations between the various algorithms involved in the Pandora reconstruction chain are minimal. Additionally, to extract the selection efficiencies, we assumed that most of the variables involved in the event selection cuts are uncorrelated. These assumptions hold true to a certain extent.

To test this result and the hypotheses that were made to obtain it, we performed an independent evaluation of the vertex performance. Using a different configuration, where only the vertex creation algorithm was cheated, we estimated the inefficiencies associated to an ``incorrect reconstruction'' of the interaction vertex. The outcome of this study, quoting the inefficiencies of the vertex reconstruction to \SI{7.0(1)}{\percent}, allowed us to independently cross-check the main work presented in this thesis: the inefficiencies observed in the main body of this work accounted for \SI{8.1(1.6)}{\percent}, compatible within 1$\sigma$ to these independent results. 

This work shows that in order to significantly improve the performance of the event reconstruction, some work has to be devoted to improving the vertex reconstruction. Additionally, it highlighted other areas of the event reconstruction where improvements are possible: the cluster creation and refinement, for example, that take place in both the first stage and the stage devoted to three-dimensional reconstruction, are not fully optimised for the ICARUS detector, and tuning the parameters involved in the reconstruction can lead to improving the performances of the event reconstruction. 

In conclusion, this work had two main objectives. The first was to validate the implementation of the cheating tools within the ICARUS event reconstruction chain, showing their strong points and highlighting the steps that require more attention. The second was to demonstrate that this approach, which targets precise physics analysis, leads to meaningful results and provides a path to the next steps in improving the event reconstruction chain. 

%% 
%%