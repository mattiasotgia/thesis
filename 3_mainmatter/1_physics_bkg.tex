% !TEX root=../main.tex

\chapter{Active and sterile neutrinos}
\label{chap:theory_introduction}

\section{Neutrinos}

Dating back to 1914, the history of neutrions began with the first paper published by Sir James Chadwick \cite{chadwickIntensityDistributionMagnetic1914}, who, investigating the phenomena of beta decays, discovered that the emitted electron energy spectra was not a single vertical emission line (delta shaped), but a continuous spectrum. 

The $\beta$ decay process, up until then, was though to be just the emission of a single electron from a neutron decaing at rest to a proton \[
    \Pn \to \Pp + \Pem,
\] and so it was expected the electron to carry all of the neutron energy. A continuous $\Pem$ spectrum broke this expectation. W. Pauli in 1930 proposed the idea that the emission of the electron occurred along the emision of another fermionic particle, way less massive --- massless even --- than the electron, carrying no electric charge \cite{pauliDearRadioactiveLadies1978} \[
    \Pn \to \Pp + \Pem + \PAGne. 
\] Though not calling this particle neutrino yet, its idea was all contained in Pauli's letter.

Enrico Fermi, a prominent scientist of that era, developerd on Pauli's idea, calling this new particle \emph{neutrino} \cite{fermiTentativoDiTeoria1934, fermiVersuchTheorieVStrahlen1934} --- from \emph{neutron}, the only chargeless particle discovered so far, aside from photons, adding the suffix \emph{-ino}, meaning smaller (and lighter). Fermi's idea was the first \emph{field theory} of quantum mechanics, suggesting that the $\beta$ decay was to be formalized as a four-fermion point interaction, involving a neutron, decaying to a proton, to produce an electron and a neutrino, \begin{equation}
    \beta^-: \quad \Pn \to \Pp + \Pem + \PAGne. \implies 
    \feynmandiagram[baseline=(current bounding box.center), horizontal=a to b, scale=0.85]{
        a [particle=$\PAGne$] -- [fermion] e,
        c [particle=$\Pn$] -- [fermion] e,
        e -- [fermion] b [particle=$\Pp$],
        e -- [fermion] d [particle=$\Pem$] 
    };
\end{equation}

Fermi's effective theory was able to explain the $\beta$ decay electron energy spectrum successfully, even preserving the angular momentum conservation. The value of the coupling measured by Fermi for this interaction, called $G_F$, was \begin{equation}
    G_F^{(\beta)} \simeq \SI{1.166e-5}{\per\giga\electronvolt\squared}, 
\end{equation} implying that this type of inteaction was extremely small, which justified calling this type of interaction \emph{weak}. Up until now however the neutrino was yet to be ``directly'' observed. It took twenty-six years of experimental efforts to actually detect the traces of this ``ghostly'' particle. The first experimental observation of electron anti-neutrinos produced by beta decays from the Savannah River reactor happened in 1956; a team led by F. Reines and C. L. Cowan observed the signature of inverse beta decay process (IB) \begin{equation}
    \PAGne + \Pp \to \Pn + \Pep
\end{equation} in a water tank, detecting the two gamma rays from proton annihilation in water with a liquid scintillator \cite{cowanDetectionFreeNeutrino1956}. 

The world of particle physics was discovering new particles very fast, and putting together the picture we today now as the Standard Model (SM) of particle physics: in the same years the electron neutrino was discovered, a team led by Carl D. Anderson and Seth Neddermeyer was discovering the muon, which they described as a heavier relative to the electron, since it showed a less prominent curvature than the electron when passed through a magnetic field; this discovery was later confirmed \cite{andersonCloudChamberObservations1936, neddermeyerNoteNatureCosmicRay1937, streetNewEvidenceExistence1937}. With the discovery of the muon, some started to question the true nature of neutrinos. In 1959 Bruno Pontecorvo examined this problem, and wondered wheter neutrions produced alongside electrons where the same as neutrinos produces alongside muons \cite{pontecorvoElectronMuonNeutrinos1991}. 


\begin{align}
    P(\PGne \rightarrow \PGne) &\simeq 1 - 4 |U_{\Pe4}|^2 (1 - |U_{\Pe4}|^2) \sin^2 \left( \frac{\Delta m^2_{41} L}{4 E_\PGn} \right) \nonumber \\
    &\qquad\qquad\qquad\equiv 1 - \sin^2 (2\theta_{ee}) \sin^2 \left( \frac{\Delta m^2_{41} L}{4 E_\PGn} \right),\\
    %
    P(\PGnGm \rightarrow \PGnGm) &\simeq 1 - 4 |U_{\PGm4}|^2 (1 - |U_{\PGm4}|^2) \sin^2 \left( \frac{\Delta m^2_{41} L}{4 E_\PGn} \right) \nonumber \\
    &\qquad\qquad\qquad\equiv 1 - \sin^2 (2\theta_{\mu\mu}) \sin^2 \left( \frac{\Delta m^2_{41} L}{4 E_\PGn} \right),\\
    %
    P(\PGnGm \rightarrow \PGne) &\simeq 4 |U_{\PGm4}|^2 |U_{\Pe4}|^2 \sin^2 \left( \frac{\Delta m^2_{41} L}{4 E_\PGn} \right) \nonumber \\
    &\qquad\qquad\qquad\equiv \sin^2 (2\theta_{\mu e}) \sin^2 \left( \frac{\Delta m^2_{41} L}{4 E_\PGn} \right).
\end{align}