% !TEX root = ../main.tex

\addchap{Introduction}

% Catchy introduction: neutrino world, something hystorical and so on...
Neutrinos are the most abundant particles in the universe. Billions of them pass through each square centimetre every second, primarily originating from our neighbouring star, the Sun. Within the Sun's core, nuclear reactions create billions of neutrinos, which travel unaltered to Earth due to their weakly interactive nature. Other sources of neutrinos include core-collapse supernovae, cosmic rays interacting within the Earth's atmosphere and nuclear reactors and accelerator complexes. 

The discovery of neutrino oscillations, and thus evidence for neutrino masses, is striking proof of physics Beyond the Standard Model (BSM). 
The generation of neutrino masses is qualitatively different from the generation of masses for any other fermionic particle in the Standard Model (SM).  
There are several possible scenarios for introducing neutrino masses in the SM. In general, the mechanism for neutrino masses would require the addition of new particle states to the SM that have never been observed. Adding these particle states would substantially modify neutrino-related observables and affect oscillation phenomenology, for example. 

Interest in this area has recently been fuelled by a series of anomalous neutrino measurements in short-baseline oscillation experiments at particle accelerators, such as the LSND and MiniBooNE collaborations, as well as in Gallium-based experiments, such as the GALLEX, SAGE and BEST collaborations, and in reactor baselines, such as the Neutrino-4 collaboration.

However, none of these short-baseline experimental anomalies proved to be definitive, even though the overall picture shows significant discrepancies with the current model. A programme with the aim of achieving a sensitivity greater than $>5\sigma$ on multiple short-baseline oscillation channels is needed to test these results and provide a comprehensive overview of these anomalies.

The Short Baseline Neutrino (SBN) programme at Fermilab comprises three detectors and is a short-baseline, multiple-oscillation-channel experimental project located along the Booster Neutrino Beam (BNB) baseline. All three detectors in the beamline are liquid argon time projection chambers (LArTPCs), exploiting the high-precision calorimetric capabilities and millimetre-scale three-dimensional tracking capabilities of such detectors to achieve unparalleled sensitivity in the search for sterile neutrinos. 

The ICARUS T600 detector acts as the SBN Far Detector (SBN-FD) at a baseline of \SI{600}{\meter}.
The location of the ICARUS detector and of the SBN Near Detector, SBND, were chosen to optimise sensitivity to neutrino oscillations and minimise the impact of flux systematics. The ICARUS detector additionally collects neutrinos coming from the Neutrino from the Main Injector (NuMI) Beam, crossing the detector ${\sim}\SI{6}{\degree}$ off-axis with respect to the direction of BNB. 
The ICARUS detector has now completed its fourth physics campaign, with the first three taking place while the SBN near detector was preparing to begin its own physics operations. With this data the collaboration has started investigating the $\PGnGm$-disappearance channel, with the simplest topologies being $1\PGm1\Pp$ and $1\PGm N\Pp$. 

% Thesis main goal, supported by nothing else than the idea
In order to improve the event selection efficiency and minimise systematic uncertainties relating to reconstruction efficiency, a thorough investigation of event reconstruction in the ICARUS TPC is necessary. 
This should be accompanied by an effort to align the signal processing and event reconstruction chains of the experiments involved in the SBN program, in view of future joint analyses.

Of all the steps involved in event processing and reconstruction, one of the most important is the construction of particle objects from the charge signal deposited on the readout wireplanes, followed by the creation of the event hierarchy (i.e. defining the primary particles originating from the interaction vertex and the interaction structure). This serves as a starting point for subsequent physics analyses. 
This process is performed by multiple tools, some taking a more ``classical'' approach to the event reconstruction, others implementing an end-to-end machine-learning-based set of tools. 

Widespread among multiple experiments employing the LArTPC technology, the more ``classical'' approach to the event reconstruction is based on a comprehensive set of tools known as Pandora. Pandora was developed to be highly modular, allowing each of the algorithms to be used in different stages of the event reconstruction chain and to be updated by new, more efficient algorithms solving the same task. This modular approach also enables algorithms that alter the performance of the reconstruction to be implemented in the reconstruction chain. This approach is key to making targeted improvements to the event reconstruction chain, which is ultimately the aim of this work. 

To be able to use these tools, a preliminary validation of the algorithms involved in the reconstruction is required; once algorithms are validated for use within the ICARUS event reconstruction, the focus will turn to demonstrating their effectiveness by conducting a thorough efficiency analysis of the TPC reconstruction chain.
This will validate the current analysis and lay the groundwork for future work, such as the future $\PGne$-appearance analysis, where these tools can be used to improve the reconstruction of shower-like particles. Reconstructing these particles is challenging due to the topology of the particle-argon interaction and the signal it produces.

% Thesis structure
The structure of this thesis follows:
\begin{itemize}
    \item \autoref{chap:theory_introduction} is devoted to introducing the theoretical framework of the Standard Model of Particle Physics, with a great interest in the physics of neutrinos, their \emph{classical} picture, the phenomenology of neutrino oscillator behaviour, and some of the anomalies driving the sterile neutrino picture. 
    \item \autoref{chap:icarus_detector} provides a detailed description of the ICARUS T600 detector, its three sub-systems (the Liquid Argon Time Projection Chamber, the light collection system, and the cosmic ray tagging system), and its physics goals, both in the context of the SBN program, which is its primary physics goal, and as a standalone experiment. 
    \item \autoref{chap:event_reconstruction} describes the major steps involved in the data acquisition, data signal processing and the software involved in the event reconstruction process for each of the ICARUS subdetectors, with a major focus on the TPC signal processing and event reconstruction. 
    \item In \autoref{chap:methods} the tools used to study the efficiency of the event reconstruction are described and validated; the methods used to perform the analysis to extract the efficiencies of the stages involved in the event reconstruction are presented. Finally some possibilities on how to improve the event reconstruction in the point where the most work is needed are presented. 
\end{itemize}

