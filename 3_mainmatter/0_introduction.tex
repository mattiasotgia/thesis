% !TEX root = ../main.tex

\addchap{Introduction}

% Catchy introduction: neutrino world, something hystorical and so on...
Neutrinos are the most abundant particles in the universe. Billions of these elusive particles pass through each square centimetre every second, primarily originating from our neighbouring star, the Sun. Within the Sun's core, they are created by nuclear reactions and then travel unaltered to the Earth due to their weakly interactive nature. Other sources of neutrinos include core-collapse supernovae, cosmic rays interacting within the Earth's atmosphere and nuclear reactors and accelerator complexes. 

The discovery of neutrino oscillations, and thus the need to include a mass term for neutrinos, is a striking proof of physics Beyond the Standard Model (BSM). 
Given the current limits on neutrino masses and their smallness compared with other Standard Model particles, several alternative mechanisms for their generation have been proposed over the years. In general, mechanisms that generate neutrino masses require the introduction of new particle states beyond the Standard Model, none of which have yet been observed. Adding these particle states would substantially modify neutrino-related observables and affect oscillation phenomenology. 

Interest in this area has recently been driven by a series of anomalous neutrino measurements in short-baseline oscillation experiments at particle accelerators, such as LSND and MiniBooNE, as well as in Gallium-based experiments, such as GALLEX, SAGE and BEST, and in reactor-based experiments, including recently Neutrino-4.

However, none of these experimental anomalies proved to be definitive, and the overall measurement scenario is far from being understood, with large inconsistencies between results derived from different channels. A programme to achieve a sensitivity larger than $5\sigma$ on multiple short-baseline oscillation channels is essential to test past measurements and provide a comprehensive overview of neutrino oscillation physics.

The Short Baseline Neutrino (SBN) programme at Fermilab was designed with this goal and is a short-baseline, multi-detector experimental project based at Fermilab and taking data along the Booster Neutrino Beam (BNB) baseline. All the detectors are liquid argon time projection chambers (LArTPCs), exploiting the high-precision calorimetric capabilities and millimetre-scale three-dimensional tracking capabilities this technique guarantees to achieve unprecedented sensitivity in the search for sterile neutrinos. 

The SBND detector operates closer to the BNB beam source, at a distance of \SI{110}{\metre} acting as the Near Detector for the SBN program; it is relevant for the SBN program since it provides a flux monitor, measuring the unoscillated neutrino spectrum.

The ICARUS T600 detector acts as the SBN Far Detector (SBN-FD) at a baseline of \SI{600}{\meter}.
The locations of the ICARUS detector and of the SBN Near Detector, SBND, were chosen to optimise sensitivity to neutrino oscillations and minimise the impact of flux systematics. The ICARUS detector is also exposed to the flux of neutrinos coming from the Neutrino from the Main Injector (NuMI) Beam, crossing the detector ${\sim}\SI{6}{\degree}$ off-axis with respect to the direction of BNB. 
The ICARUS detector has now completed its fourth physics campaign, with the first data-taking campaign conducted with the full SBN detector suite fully operational and in physics mode. With a fraction of these data corresponding to the second data-taking campaign, the collaboration has started investigating the $\PGnGm$-disappearance channel, analysing the two among the simplest topologies, i.e., $1\PGm1\Pp$ and $1\PGm N\Pp$ events. 

% Thesis main goal, supported by nothing else than the idea
In order to improve the event selection efficiency and minimise systematic uncertainties relating to reconstruction efficiency, a thorough investigation of event reconstruction in the ICARUS TPC is necessary. 
This should be accompanied by an effort to align the signal processing and event reconstruction chains of the experiments involved in the SBN program, in view of future joint analyses.

Among the steps involved in event processing and reconstruction, one of the most important is the creation of particle objects from the charge signal deposited on the readout wireplanes, followed by the construction of the event hierarchy that entails the identification of the interaction structure and especially primary particles originating directly from the interaction vertex. This serves as a starting point for subsequent event processing steps and analyses. 
This process is performed using topological and calorimetric information by two separate set of tools: the former following a machine-learning-based approach and the latter relying on traditional rule-based, physics-informed methods.

Widespread among multiple experiments employing the LArTPC technology, the traditional, physics-informed approach to the event reconstruction is based on a comprehensive set of algorithms known as Pandora. Pandora was developed to be highly modular, allowing each of the algorithms to be used in different stages of the event reconstruction chain and to be updated by new, more efficient algorithms solving the same task. This modular approach also permits the introduction of algorithms that alter the reconstruction to allow usage of Monte Carlo information for simulated events. This approach is key to making targeted improvements to the event reconstruction chain, which is ultimately the aim of this work. 

To be able to use these tools, a preliminary validation of the algorithms involved in the reconstruction and their altered, truth-based version is required; once algorithms were validated for use within the ICARUS event reconstruction, the focus became their joint usage with a targeted event selection to draw an estimate of the efficiencies of the main reconstruction steps.
This will allow targeted improvements to reconstruction stages, thereby laying the ground for future work, including but not limited to the upcoming $\PGne$-appearance analysis, where refined reconstruction tools/methods/algorithms can provide improved shower-like particle reconstruction.
Reconstructing this type of interaction is challenging due to the complex final state topology it produces.

% Thesis structure
The outline of the thesis is the following:
\begin{itemize}
    \item \autoref{chap:theory_introduction} is devoted to introducing the theoretical and experimental framework of the Standard Model of Particle Physics, and provides an overview of the present status of neutrino physics, including their classical picture, the phenomenology of neutrino oscillator behaviour, and the main anomalies driving the current sterile neutrino scenario.
    \item \autoref{chap:icarus_detector} provides a detailed description of the ICARUS T600 detector, its three sub-systems (the Time Projection Chamber, the light collection system, and the cosmic ray tagging system), and its main physics goals, both in the context of the SBN program and as a standalone experiment. 
    \item \autoref{chap:event_reconstruction} describes the major steps involved in the data acquisition and data signal processing with details on the tools employed to reconstruct physical information on the measured interactions for each ICARUS subsystem and especially on the TPC signal processing and event reconstruction. 
    \item In \autoref{chap:methods} the specific set of tools employed to study the efficiency  of the event reconstruction is described and validated; the strategy used to perform the analysis to extract the efficiencies of the stages involved in the event reconstruction is presented. Finally, based on the outcome of this study, the next steps to improve specific stages of the reconstruction chain are illustrated.
\end{itemize}

