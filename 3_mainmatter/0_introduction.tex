% !TEX root = ../main.tex

\addchap{Introduction}

% Catchy introduction: neutrino world, something hystorical and so on...
Neutrinos are the most abundant particle in the universe, with billions of neutrinos passing trough each square centimeter each second, primarily coming from our neighbour star, the Sun: within nuclear reactions inside the Sun's core, billions of neutrino are created, and, due to their weakly interactive nature they travel unaltered to Earth. Other neutrino sources are also core-collapse supernovae, interacting cosmic ray within Earth's atmosphere, and, nonetheless, nuclear reactors and accelerator complexes. 

The discovery of neutrino oscillations, hence the evidence for neutrino masses, is a striking proof of Beyond the Standard Model (BSM) physics. 
Generating neutrino masses is qualitatively different from generating masses for any other fermionic particle in the Standard Model (SM).  
Several are the possible scenarios for introducing neutrino masses in the SM: in general, the mechanism of neutrino masses would require addition of new particle state to the SM that have never been observed. The addition of this particle states would modify substantially neutrino-related observables, and would have effects, for example, on oscillation phenomenology. 



A common thread across most of these mass models is questioning the number of neutrino species. This is supported by different experimental anomalies at accelerating complexes, like for the LSND and MiniBooNE collaborations, at Gallium-based experiments, like GALLEX, SAGE and BEST collaborations, and at reactors baselines, like for the Neutrino-4 collaboration. None of these short-baseline experimental anomalies hovewer proved to be definitive, even if the global picture shows a strong tension with the current model. An individual program aiming at a $>5\sigma$ sensitivity, on multiple short-baseline oscillation channels experiment is needed to test these results and draw a definitive picture for those short-baseline experimenal anomalies. 

The Short Baseline Neutrino (SBN) program at Fermilab is a three detector, short-baseline, multiple oscillation channel experimental effort, located along the Booster Neutrino Beam (BNB) baseline, aiming at exploring the same phase space of previous experiments. 
The ICARUS T600 detector acts as the SBN Far Detector (SBN-FD) at a baseline of \SI{600}{\meter}.
The location of both the ICARUS detector and of the SBN Near Detector, SBND, where chosen to optimize neutrino oscillation sensitivity and minimize the impact of flux systematics. 

% Thesis main goal, supported by nothing else than the idea


% Thesis structure
The thesis structure is as follows\begin{itemize}
    \item \autoref{chap:theory_introduction} is dedicated to introducing the theorethical framework of the Standard Model of Particle Physics, with a great interest on the physics of neutrinos, their \emph{classical} picture, the phenomenology of neutrino oscillator behaviour and some of the anomalies driving the sterile neutrino picture. 
    \item \autoref{chap:icarus_detector} tries to get a detailed description of the ICARUS T600 detector, its three sub-systems, the Liquid Argon Time Projection Chamber, the light collection system and the cosmic ray tagging system, and its role in the Fermilab Short Baseline Neutrino Program. 
    \item \autoref{chap:event_reconstruction} is dedicated to an overview of the event reconstruction in all the T600 sub-detectors, with a primary focus on the TPC event reconstruction. 
\end{itemize}

