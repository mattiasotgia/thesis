% !TEX root = ../main.tex

\addchap{Introduction}

% Catchy introduction: neutrino world, something hystorical and so on...
Neutrinos are the most abundant particle in the universe, with billions of neutrinos passing trough each square centimeter each second, primarily coming from our neighbour star, the Sun: within nuclear reactions inside the Sun's core, billions of neutrino are created, and, due to their weakly interactive nature they travel unaltered to Earth. Other neutrino sources are also core-collapse supernovae, interacting cosmic ray within Earth's atmosphere, and, nonetheless, nuclear reactors and accelerator complexes. 

The discovery of neutrino oscillations, hence the evidence for neutrino masses, is a striking proof of Beyond the Standard Model (BSM) physics. 
Generating neutrino masses is qualitatively different from generating masses for any other fermionic particle content of the Standard Model (SM).  
Several are the possible scenarios for introducing neutrino masses in the SM: in general, the mechanism of neutrino masses would require addition of new particle state to the SM that have never been observed. The addition of this particle states would modify substantially neutrino-related observables, and would have effects, for example, on oscillation phenomenology. 

Interest in this direction has been fanned, more recently, by a series of neutrino anomalous measurement at short-baseline oscillation experiments, at accelerating complexes, like the LSND and MiniBooNE collaborations, at Gallium-based experiments, like GALLEX, SAGE and BEST collaborations, and at reactors baselines, like the Neutrino-4 collaboration.

None of these short-baseline experimental anomalies, hovewer, proved to be definitive, even if the global picture shows a strong tension with the current model. An individual program aiming at a $>5\sigma$ sensitivity, on multiple short-baseline oscillation channels experiment is needed to test these results and draw a complete picture for these short-baseline experimental anomalies. 

The Short Baseline Neutrino (SBN) program at Fermilab is a three detector, short-baseline, multiple oscillation channel experimental effort, located along the Booster Neutrino Beam (BNB) baseline. All the three detectors in the beamline are Liquid Argon Time Projection Chambers (LArTPC(s)), exploiting on the high precision calorimetric power and mm-scale three dimensional tracking capabilities of such detectors to archive unprecedented sensitivity on the sterile neutrino search. 

The ICARUS T600 detector acts as the SBN Far Detector (SBN-FD) at a baseline of \SI{600}{\meter}.
The location of both the ICARUS detector and of the SBN Near Detector, SBND, where chosen to optimize neutrino oscillation sensitivity and minimize the impact of flux systematics. Among the Booster Neutrino Beam, the ICARUS T600 detector is also on the baseline of the Neutrino from the Main Injector (NuMI) Beam, crossing the detector $\SI{6}{\degree}$ off-axis wih respect to the detector principal axis. 
The ICARUS detector is now finishing its fourth physics run, three of which were done while the SBN near detector was preparing to start its physic operation. With all this data the collaboration has started to look into $\PGnGm$-disappearance studies, with the simplest topologies being $1\PGm1\Pp$ and $1\PGm N\Pp$. 

% Thesis main goal, supported by nothing else than the idea
In order to reduce the systematic uncertainties related to the reconstruction efficiency, a detailed study of the event reconstruction inside the ICARUS TPC is needed, alongside an effort to align the ICARUS and SBND detectors signal processing and event reconstruction chain, in view of the future SBN joint analysis. 

Of all steps involved in the event processing and reconstruction, one of great importance is related to the particle objects building from the signals left on the wireplanes, and the subsequent event hierarchy creation (that is defining which are the primary particles originating from the interaction vertex and the interaction \emph{structure}), which is the centerpiece of many further analysis. 
This process is performed by a set of algorithms shared across the LArTPC technology detectors. The common framework is based on the Pandora Patter Finding Algorithm software. This feature, alongside the various algorithms suited for the reconstruction, a set of tools that can be used to perform studies on the reconstruction efficiency, previously unused by the ICARUS collaboration. 

The goal of this thesis is to validate this set of tools for further use in the ICARUS collaboration, and show their power by performing a detailed efficiency analysis of the TPC reconstruction chain. 
This will likely serve both as a validation for the current analysis, as wall as a foundation for later works --- such as the future $\PGne$-appearance analysis --- where these tools can be used to validate the reconstruction for the shower-like particles, where the reconstruction hit a big wall due to the particle-argon interaction topology and the signal that it produces. 

% Thesis structure
The thesis structure is as follows\todo[red]{This section need a lot of work to be complete... add the correct sections of the work, and a better description}
\begin{itemize}
    \item \autoref{chap:theory_introduction} is devoted to introducing the theoretical framework of the Standard Model of Particle Physics, with a great interest on the physics of neutrinos, their \emph{classical} picture, the phenomenology of neutrino oscillator behaviour and some of the anomalies driving the sterile neutrino picture. 
    \item \autoref{chap:icarus_detector} tries to get a detailed description of the ICARUS T600 detector, its three sub-systems, the Liquid Argon Time Projection Chamber, the light collection system and the cosmic ray tagging system, and its role in the Fermilab Short Baseline Neutrino Program. 
    \item \autoref{chap:event_reconstruction} is dedicated to an overview of the event reconstruction in all the T600 sub-detectors, with a primary focus on the TPC event reconstruction. I will show briefly the information that each sub-detector is capable of collecting and how this is used in the offline reconstruction. 
    \item \autoref{chap:methods} is the focus of my research activity. There I will introduce the tools I developed and the techniques I used to analyze the data. 
    \item \autoref{chap:conclusions} show the 
\end{itemize}

