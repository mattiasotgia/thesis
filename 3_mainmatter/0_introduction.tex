% !TEX root = ../main.tex

\addchap{Introduction}

% Catchy introduction: neutrino world, something hystorical and so on...
Neutrinos are the most abundant particles in the universe. Billions of them pass through each square centimetre every second, primarily originating from our neighbouring star, the Sun. Within the Sun's core, nuclear reactions create billions of neutrinos, which travel unaltered to Earth due to their weakly interactive nature. Other sources of neutrinos include core-collapse supernovae, cosmic rays interacting within the Earth's atmosphere and nuclear reactors and accelerator complexes. 

The discovery of neutrino oscillations, and thus evidence for neutrino masses, is striking proof of physics Beyond the Standard Model (BSM). 
The generation of neutrino masses is qualitatively different from the generation of masses for any other fermionic particle in the Standard Model (SM).  
There are several possible scenarios for introducing neutrino masses in the SM. In general, the mechanism for neutrino masses would require the addition of new particle states to the SM that have never been observed. Adding these particle states would substantially modify neutrino-related observables and affect oscillation phenomenology, for example. 

Interest in this area has recently been fuelled by a series of anomalous neutrino measurements in short-baseline oscillation experiments at particle accelerators, such as the LSND and MiniBooNE collaborations, as well as in Gallium-based experiments, such as the GALLEX, SAGE and BEST collaborations, and in reactor baselines, such as the Neutrino-4 collaboration.

However, none of these short-baseline experimental anomalies proved to be definitive, even though the overall picture shows significant discrepancies with the current model. A programme with the aim of achieving a sensitivity greater than $>5\sigma$ on multiple short-baseline oscillation channels is needed to test these results and provide a comprehensive overview of these anomalies.

The Short Baseline Neutrino (SBN) programme at Fermilab comprises three detectors and is a short-baseline, multiple-oscillation-channel experimental project located along the Booster Neutrino Beam (BNB) baseline. All three detectors in the beamline are liquid argon time projection chambers (LArTPCs), exploiting the high-precision calorimetric capabilities and millimetre-scale three-dimensional tracking capabilities of such detectors to achieve unparalleled sensitivity in the search for sterile neutrinos. 

The ICARUS T600 detector acts as the SBN Far Detector (SBN-FD) at a baseline of \SI{600}{\meter}.
The location of both the ICARUS detector and of the SBN Near Detector, SBND, was chosen to optimise neutrino oscillation sensitivity and minimise the impact of flux systematics. Among the Booster Neutrino Beam, the ICARUS T600 detector is also on the baseline of the Neutrino from the Main Injector (NuMI) Beam, crossing the detector $\SI{6}{\degree}$ off-axis with respect to the detector principal axis. 
The ICARUS detector has now finished its fourth physics campaign, three of which were done while the SBN near detector was preparing to start its physics operation. With all this data the collaboration has started to look into $\PGnGm$-disappearance studies, with the simplest topologies being $1\PGm1\Pp$ and $1\PGm N\Pp$. 

% Thesis main goal, supported by nothing else than the idea
In order to reduce the systematic uncertainties related to the reconstruction efficiency, a detailed study of the event reconstruction inside the ICARUS TPC is needed, alongside an effort to align the ICARUS and SBND detectors signal processing and event reconstruction chain, in view of the future SBN joint analysis. 

Of all steps involved in the event processing and reconstruction, one of great importance is related to the particle objects building from the signals left on the wireplanes and the subsequent event hierarchy creation (that is, defining which are the primary particles originating from the interaction vertex and the interaction \emph{structure}), which is the centrepiece of many further analyses. 
This process is performed by a set of algorithms shared across the LArTPC technology detectors. The common framework is based on the Pandora Patter Finding Algorithm software. This features, alongside the various algorithms suited for the reconstruction, a set of tools that can be used to perform studies on the reconstruction efficiency, previously unused by the ICARUS collaboration. 

The aim of this thesis is twofold: first, to validate this set of tools for further use in the ICARUS collaboration, and second, to demonstrate their effectiveness by conducting a thorough efficiency analysis of the TPC reconstruction chain.
This will serve as validation for the current analysis and lay the groundwork for future work, such as the future $\PGne$-appearance analysis, where these tools can be used to validate the reconstruction of shower-like particles. The reconstruction of these particles is difficult due to the topology of the particle-argon interaction and the signal it produces.

% Thesis structure
The thesis structure is as follows
\begin{itemize}
    \item \autoref{chap:theory_introduction} is devoted to introducing the theoretical framework of the Standard Model of Particle Physics, with a great interest on the physics of neutrinos, their \emph{classical} picture, the phenomenology of neutrino oscillator behaviour, and some of the anomalies driving the sterile neutrino picture. 
    \item \autoref{chap:icarus_detector} tries to get a detailed description of the ICARUS T600 detector, its three sub-systems (the Liquid Argon Time Projection Chamber, the light collection system, and the cosmic ray tagging system), and its role in the Fermilab Short Baseline Neutrino Program. 
    \item \autoref{chap:event_reconstruction} is dedicated to an overview of the complete chain the collected data follows, from when it is acquired by the DAQ system to the processing stages allowing this data to be informative. It focuses on how signals from all sub-detectors is reconstructed, with a primary focus on the TPC event reconstruction. 
    \item \autoref{chap:methods} \dots
    \item \autoref{chap:conclusions} \dots
\end{itemize}

